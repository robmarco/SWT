%
% Frontmatter - Introducción. Los miembros del tribunal que juzgan los PFC's tienen muchas 
% 				más memorias que leer, por lo que agradecerán cualquier detalle que permita 
%				facilitarles la vida. En este sentido, realizar una pequeña introducción,
%				comentar la organización, estructura de la memoria y resumir brevemente 
%				cada capítulo puede ser una buena práctica que permita al lector centrarse 
%				fácilmente en la parte que más le interesa.
%
% 				Contenido de prueba
%

\chapter[Introducción]{
	Introducción
}

	En Canarias existen numerosos equipos de natación que, a lo largo de los años, han proporcionado numerosos éxitos a nivel regional y nacional. 

	Tradicionalmente, en la gestión deportiva los procesos se realizan usando métodos analógicos y, en los que en muchos casos, la información no fluye sobre los equipos como debiese. Así mismo, en un intento de mejorar en el control deportivo, esta información se pasa a digital y se realizan informes, dejando de lado detalles debidos al coste de la duplicación de la información.

	Cada vez más, estos equipos están formados por técnicos cualificados (muchos de ellos Licenciados en Ciencias de la Actividad Física y Deporte), lo que hace que se haya implantado una metodología de trabajo común: desde el control de nadadores, pasando por la gestión de entrenamientos diarios y competiciones. Este hecho permite que se puedan estandarizar y plasmar en una aplicación los flujos de información incluidos en el proceso.

	Además, en condiciones normales, los equipos tienen sedes físicas donde almacenan toda la información de la misma y que sirve como punto de reunión. El problema radica en que cuando los equipos realizan desplazamientos, no toda la información puede ser llevada encima. Se produce por tanto una disminución en términos de accesibilidad a la misma.

	Por tanto, con el objetivo de poder gestionar toda la información referente a los nadadores, se requiere desarrollar un sistema informático capaz de realizar determinadas tareas de control en los flujos de información - pertenecientes a las metodologías de trabajo de los entrenadores de natación - tales como: gestión de fichas de nadadores, entrenamientos, competiciones y test preparatorios. Con ello, se aumentará la capacidad de análisis de información y posterior aplicación en futuras planificaciones de entrenamientos. Al estar en un entorno web, supone una ventaja en cuanto a accesibilidad, evitando con ello los problemas de portabilidad encontrados en aplicaciones de escritorio.

\section*{Objetivo del proyecto} % (fold)
	\label{sec:objetivo_del_proyecto}
	
	El objetivo fundamental  del proyecto es desarrollar un software de gestión deportiva para entrenadores de natación. Las tareas que se van a abordar para realizar dicha gestión son las siguientes:
	
	\begin{enumerate}
		\item {{\bf Gestión de acceso de entrenadores al sistema}. Se controla la fase de registro y autenticación al sistema, así como la configuración del perfil de los entrenadores.}
		\item {{\bf Gestión de fichas de nadadores}. Nadadores pertenecientes a la sección que maneja un entrenador.}
		\item {{\bf Gestión de entrenamientos}. Posibilidad de controlar la asistencia a los entrenamientos; subir al sistema la planificación de la temporada; y administrar las tablas de entrenamientos diarias.}
		\item {{\bf Gestión de competiciones}. Establecimiento de un calendario deportivo, donde se almacenan todos los eventos de una temporada, y control de los resultados de cada uno de los nadadores en las competiciones.}
		\item {{\bf Gestión de test preparatorios}. Control de cada una de las pruebas realizadas por los entrenadores.}
		\item {{\bf Diario del entrenador}. Capacidad del sistema para anotar incidencias y observaciones que el entrenador crea conveniente.}
		\item {{\bf Estadísticas}. Informes estadísticos de los entrenamientos, competiciones y test de cada uno de los nadadores.}
	\end{enumerate}
	
	El software se desarrollará para plataforma web, diseñando las distintas interfaces de la aplicación y donde el usuario destino podrá gestionar cada una de las tareas mencionadas.