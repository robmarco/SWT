\documentclass[a4paper,oneside,11pt]{book}
	
\usepackage[spanish]{babel}
\usepackage[utf8]{inputenc}

% -------------------------------------------------------------------------

% Para incluir graficos en JPG => compilar con pdflatex.
%\usepackage[pdftex]{graphicx}
% Para incluir graficos EPS => compilar con latex.
\usepackage[dvips]{graphicx}
% Para escribir en color cuando compilamos con pdflatex
\usepackage[pdftex,usenames,dvipsnames]{color}

% -------------------------------------------------------------------------

% Paquete fancyhdr -> Para modificar la cabecera y pie de paginas.
% http://tug.ctan.org/tex-archive/macros/latex/contrib/fancyhdr/
\usepackage{fancyhdr}
\pagestyle{fancy}
\fancyhf{}
\fancyhf[HR]{\thepage}
\fancyhf[HL]{\nouppercase\rightmark}
\headwidth=16cm

% Ajuste de margenes
\usepackage{anysize}
\marginsize{3cm}{2cm}{2cm}{2cm}	% izquierda, derecha, arriba, abajo
\parskip=6pt	% Personalizamos la separacion entre parrafos.
\parindent=10pt	% Personalizamos el identado en la primera linea del nuevo parrafo.
\setcounter{secnumdepth}{3}	% Numero maximo de niveles de profundidad en las secciones.

% -------------------------------------------------------------------------

% Package booktabs -> Para mejorar el aspecto de las tablas o cuadros.
% http://www.ctan.org/tex-archive/macros/latex/contrib/booktabs/
\usepackage{booktabs}

% Package rotating -> Para poder girar las tablas y dibujarlas a lo largo
% del folio en vez de a lo ancho.
\usepackage{rotating}

% Packages multicol y multirow, para manejar tablas de filas y columnas multiples.
\usepackage{multicol}
\usepackage{multirow}

\usepackage{tabularx}
% -------------------------------------------------------------------------

\title{Desarrollo Aplicación Web de Gestión Deportiva para Entrenadores de Natación}
\author{Roberto Marco Sánchez}
\date{\today}

\begin{document}
	\maketitle
	% FRONTMATTER: TOC, LOF, LOT y descripcion/organizacion de la memoria.
	\frontmatter

	\tableofcontents
	\listoffigures
	\listoftables
	%
% Frontmatter - Introducci�n. Los miembros del tribunal que juzgan los PFC's tienen muchas m�s memorias que leer, por lo que
%	agradecer�n cualquier detalle que permita facilitarles la vida. En este sentido, realizar una peque�a introducci�n,
%	comentar la organizaci�n y estructura de la memoria y resumir brevemente cada cap�tulo puede ser una buena pr�ctica
%	que permita al lector centrarse f�cilmente en la parte que m�s le interesa.
%

\chapter[Introducci�n]{
	Introducci�n
}

La introducci�n es lo primero que se lee, pero habitualmente lo �ltimo que se escribe.
Pues su redacci�n depende de c�mo se hayan escrito todas las otras secciones. Normalmente la introducci�n incluye una descripci�n muy general del proyecto y termina con un desglose del contenido de la memoria.

One for all and all for one, Muskehounds are always ready. One for all and all for one, helping everybody. One for all and all for one, it's a pretty story. Sharing everything with fun, that's the way to be. One for all and all for one, Muskehounds are always ready. One for all and all for one, helping everybody. One for all and all for one, can sound pretty corny. If you've got a problem chum, think how it could be.

Hong Kong Phooey, number one super guy. Hong Kong Phooey, quicker than the human eye. He's got style, a groovy style, and a car that just won't stop. When the going gets tough, he's really rough, with a Hong Kong Phooey chop (Hi-Ya!). Hong Kong Phooey, number one super guy. Hong Kong Phooey, quicker than the human eye. Hong Kong Phooey, he's fan-riffic!

Ten years ago a crack commando unit was sent to prison by a military court for a crime they didn't commit. These men promptly escaped from a maximum security stockade to the Los Angeles underground. Today, still wanted by the government, they survive as soldiers of fortune. If you have a problem and no one else can help, and if you can find them, maybe you can hire the A-team.

This is my boss, Jonathan Hart, a self-made millionaire, he's quite a guy. This is Mrs H., she's gorgeous, she's one lady who knows how to take care of herself. By the way, my name is Max. I take care of both of them, which ain't easy, 'cause when they met it was MURDER!

Top Cat! The most effectual Top Cat! Who's intellectual close friends get to call him T.C., providing it's with dignity. Top Cat! The indisputable leader of the gang. He's the boss, he's a pip, he's the championship. He's the most tip top, Top Cat.

%
% SECCION
%
\subsection*{Estructura de la memoria}


In a dolor sed odio eleifend varius. Nam ullamcorper. Curabitur ut erat vulputate nisi molestie tempus. Sed aliquam rutrum odio. In mollis. Fusce consectetuer lorem nec diam. Sed mollis lacinia purus. Curabitur feugiat hendrerit neque. Quisque auctor laoreet diam. Curabitur sit amet nisi. Fusce velit massa, dignissim quis, bibendum eget, vehicula mattis, leo. Morbi auctor leo sit amet nibh. Lorem ipsum dolor sit amet, consectetuer adipiscing elit. Nullam enim. Pellentesque hendrerit, augue non vulputate semper, sem lorem pharetra nibh, sit amet egestas massa diam ac augue. In dui nulla, egestas nec, pulvinar suscipit, tincidunt ornare, nisi. Duis tristique tortor quis magna. Vestibulum faucibus lorem nec neque. Sed nec nibh. Nunc condimentum. Maecenas neque. Nullam pretium est non risus. Etiam gravida. Maecenas nisl. Fusce pharetra odio in tortor. Integer orci turpis, interdum eget, vulputate sed, tristique a, metus. Duis vitae dui quis lectus pretium aliquam. Praesent quam.

Aliquam sed orci. Cras adipiscing nisl quis pede. Ut rhoncus. Donec viverra laoreet purus. Phasellus nulla. Vivamus eget eros. In mollis aliquam orci. Proin ullamcorper. Nullam sollicitudin vestibulum lorem. Nunc malesuada sagittis augue. Donec tellus velit, dapibus a, aliquam ac, tincidunt id, lectus. 


\paragraph*{Cap�tulo 1.}
Phasellus tempor velit nec velit. Proin vitae dui a sapien commodo blandit. Etiam aliquam, sapien vitae fringilla venenatis, lectus sem accumsan orci, eget blandit orci odio et magna. Quisque malesuada, eros vel tempus eleifend, velit enim porttitor sem, eget consequat nulla neque et sapien. Morbi leo. Sed vestibulum lacus. Fusce ut lacus. Phasellus pellentesque pede eu eros. Duis turpis felis, eleifend ut, semper ac, porta nec, sem. Praesent odio. Sed laoreet mollis purus. Praesent vestibulum, velit ut mollis aliquam, quam lectus varius urna, sed ultricies erat nisl ac tortor. Vivamus tempor mauris sit amet nulla. Integer venenatis. Integer sagittis euismod ante. Suspendisse at elit. Duis eget purus nec pede adipiscing auctor. Proin ac est.

\paragraph*{Cap�tulo 2.}
Proin condimentum. Maecenas sodales. In ornare nunc a leo. Nam sit amet ligula. Nunc quis urna ac metus imperdiet lobortis. Sed quis ligula. Maecenas blandit pede. Donec lacinia rutrum ligula. Vivamus in metus vel elit pharetra molestie.

\paragraph*{Cap�tulo 3.}
Nullam ante lorem, placerat et, egestas nec, pellentesque non, sapien. Donec semper, felis id posuere faucibus, nibh ipsum tincidunt quam, et varius ipsum odio ac neque. In tincidunt dignissim diam. Sed lacus lorem, ornare ut, eleifend vel, pellentesque tempus, augue. Duis eu magna. Mauris libero ante, porttitor vel, lobortis a, mollis ac, sem. Nunc at lectus. Integer ac libero a nisl dignissim mollis. Donec velit neque, vestibulum eget, pulvinar vel, malesuada ut, nisi. Praesent congue tempus quam. Cum sociis natoque penatibus et magnis dis parturient montes, nascetur ridiculus mus.

\paragraph*{Cap�tulo 4.}
Mauris ut odio. Nulla accumsan. Morbi condimentum fermentum purus. Pellentesque habitant morbi tristique senectus et netus et malesuada fames ac turpis egestas. Nunc dignissim, neque eget convallis pretium, diam tortor fringilla lacus, a laoreet nisl metus eu magna. Cras ut lectus. Etiam accumsan feugiat elit.

\paragraph*{Cap�tulo ...}
Donec a pede. Proin dolor. Ut nunc ligula, tempor id, ornare sit amet, aliquam et, nibh. In mollis iaculis pede. Vivamus gravida orci eu nisl. Sed nibh sem, consequat at, iaculis non, placerat in, ligula. Praesent id nisi. Nunc pellentesque justo non libero. Sed quis est sit amet purus lobortis blandit. Sed arcu justo, rhoncus condimentum, ullamcorper iaculis, viverra et, nisl.

\paragraph*{Cap�tulo N.}
Fusce luctus gravida leo. Nullam dignissim arcu ac risus hendrerit rhoncus. Aliquam erat volutpat. Ut mollis, mauris non aliquam luctus, nulla sem aliquam tellus, in consequat augue odio in urna.

%
% SECCION
%
\subsection*{Estructura de la memoria}


In a dolor sed odio eleifend varius. Nam ullamcorper. Curabitur ut erat vulputate nisi molestie tempus. Sed aliquam rutrum odio. In mollis. Fusce consectetuer lorem nec diam. Sed mollis lacinia purus. Curabitur feugiat hendrerit neque. Quisque auctor laoreet diam. Curabitur sit amet nisi. Fusce velit massa, dignissim quis, bibendum eget, vehicula mattis, leo. Morbi auctor leo sit amet nibh. Lorem ipsum dolor sit amet, consectetuer adipiscing elit. Nullam enim. Pellentesque hendrerit, augue non vulputate semper, sem lorem pharetra nibh, sit amet egestas massa diam ac augue. In dui nulla, egestas nec, pulvinar suscipit, tincidunt ornare, nisi. Duis tristique tortor quis magna. Vestibulum faucibus lorem nec neque. Sed nec nibh. Nunc condimentum. Maecenas neque. Nullam pretium est non risus. Etiam gravida. Maecenas nisl. Fusce pharetra odio in tortor. Integer orci turpis, interdum eget, vulputate sed, tristique a, metus. Duis vitae dui quis lectus pretium aliquam. Praesent quam.

Aliquam sed orci. Cras adipiscing nisl quis pede. Ut rhoncus. Donec viverra laoreet purus. Phasellus nulla. Vivamus eget eros. In mollis aliquam orci. Proin ullamcorper. Nullam sollicitudin vestibulum lorem. Nunc malesuada sagittis augue. Donec tellus velit, dapibus a, aliquam ac, tincidunt id, lectus. 


\paragraph*{Cap�tulo 1.}
Phasellus tempor velit nec velit. Proin vitae dui a sapien commodo blandit. Etiam aliquam, sapien vitae fringilla venenatis, lectus sem accumsan orci, eget blandit orci odio et magna. Quisque malesuada, eros vel tempus eleifend, velit enim porttitor sem, eget consequat nulla neque et sapien. Morbi leo. Sed vestibulum lacus. Fusce ut lacus. Phasellus pellentesque pede eu eros. Duis turpis felis, eleifend ut, semper ac, porta nec, sem. Praesent odio. Sed laoreet mollis purus. Praesent vestibulum, velit ut mollis aliquam, quam lectus varius urna, sed ultricies erat nisl ac tortor. Vivamus tempor mauris sit amet nulla. Integer venenatis. Integer sagittis euismod ante. Suspendisse at elit. Duis eget purus nec pede adipiscing auctor. Proin ac est.

\paragraph*{Cap�tulo 2.}
Proin condimentum. Maecenas sodales. In ornare nunc a leo. Nam sit amet ligula. Nunc quis urna ac metus imperdiet lobortis. Sed quis ligula. Maecenas blandit pede. Donec lacinia rutrum ligula. Vivamus in metus vel elit pharetra molestie.

\paragraph*{Cap�tulo 3.}
Nullam ante lorem, placerat et, egestas nec, pellentesque non, sapien. Donec semper, felis id posuere faucibus, nibh ipsum tincidunt quam, et varius ipsum odio ac neque. In tincidunt dignissim diam. Sed lacus lorem, ornare ut, eleifend vel, pellentesque tempus, augue. Duis eu magna. Mauris libero ante, porttitor vel, lobortis a, mollis ac, sem. Nunc at lectus. Integer ac libero a nisl dignissim mollis. Donec velit neque, vestibulum eget, pulvinar vel, malesuada ut, nisi. Praesent congue tempus quam. Cum sociis natoque penatibus et magnis dis parturient montes, nascetur ridiculus mus.

\paragraph*{Cap�tulo 4.}
Mauris ut odio. Nulla accumsan. Morbi condimentum fermentum purus. Pellentesque habitant morbi tristique senectus et netus et malesuada fames ac turpis egestas. Nunc dignissim, neque eget convallis pretium, diam tortor fringilla lacus, a laoreet nisl metus eu magna. Cras ut lectus. Etiam accumsan feugiat elit.

\paragraph*{Cap�tulo ...}
Donec a pede. Proin dolor. Ut nunc ligula, tempor id, ornare sit amet, aliquam et, nibh. In mollis iaculis pede. Vivamus gravida orci eu nisl. Sed nibh sem, consequat at, iaculis non, placerat in, ligula. Praesent id nisi. Nunc pellentesque justo non libero. Sed quis est sit amet purus lobortis blandit. Sed arcu justo, rhoncus condimentum, ullamcorper iaculis, viverra et, nisl.

\paragraph*{Cap�tulo N.}
Fusce luctus gravida leo. Nullam dignissim arcu ac risus hendrerit rhoncus. Aliquam erat volutpat. Ut mollis, mauris non aliquam luctus, nulla sem aliquam tellus, in consequat augue odio in urna.



	
	% MAINMATTER: El contenido, capitulo a capitulo, de la memoria.
	\mainmatter

	%\chapter { Captura de requisitos \label{ch:captura_requisitos}}

En este apartado se describe de forma detallada cada una de las funcionalidades que pueden componer el proyecto, con el fin de guiar el desarrollo hacia el sistema correcto. El proceso estar� apoyado en una lista de caracter�sticas y en el modelo del dominio.

%	-------------------------------------
% 	Secci�n para lista de caracter�sticas
%	-------------------------------------
\section{Lista de Caracter�sticas} % (fold)
	\label{sec:lista_de_caracter�sticas}

	Cada caracter�stica tiene un nombre corto y una breve explicaci�n, informaci�n suficiente para poder hablar de ella durante la planificaci�n del producto. Cara caracter�stica tiene tambi�n un conjunto de valores de planificaci�n que son incluidos: 

	\begin{itemize}
		\item{Prioridad con la que se ir�n desarrollando (muy alta, alta, media, baja o muy baja)}
		\item{Estado (aceptado, planificado, en desarrollo, finalizada postergada o rechazada)}
		\item{Coste estimado de implementaci�n (en t�rminos de tipos de recursos y horas-persona)}
		\item{Nivel de riesgo asociado a la implementaci�n de la caracter�stica (cr�tico, significativo o rutinario)}
	\end{itemize}
	
	Las categor�as son: \\
		Gesti�n de entrenadores\\
			Acceso y Autentificaci�n\\
		Gesti�n de nadadores\\
		Gesti�n de entrenamientos\\
			Planificaci�n\\
			Tabla de entrenamientos\\
		Gesti�n de competiciones\\
		Diario del entrenador\\

\subsection{En relaci�n a la gesti�n de entrenadores} % (fold)

	\begin{center}
	\begin{tabulary}{15cm}{|L|}
		\hline
			\bf{LC-A1. Acceso y autentificaci�n} \\
		\hline
			Esto es un ejemplo de descripci�n de la caracter�stica seleccionada donde se ocupa m�s de la cuenta debido a que es asi.\\
		\hline
			\it{Prioridad} \\
		\hline
			\it{Estado} \\
		\hline
			\it{Coste} \\
		\hline
			\it{Riesgo} \\
		\hline
	\end{tabulary}
	\end{center}
	
% subsection en_relaci�n_a_los_entrenadores (end)
	
% section lista_de_caracter�sticas (end)



	\chapter { Captura de requisitos \label{ch:captura_requisitos}}

En este apartado se describe de forma detallada cada una de las funcionalidades que pueden componer el proyecto, con el fin de guiar el desarrollo hacia el sistema correcto. El proceso estar� apoyado en una lista de caracter�sticas y en el modelo del dominio.

%	-------------------------------------
% 	Secci�n para lista de caracter�sticas
%	-------------------------------------
\section{Lista de Caracter�sticas} % (fold)
	\label{sec:lista_de_caracter�sticas}

	Cada caracter�stica tiene un nombre corto y una breve explicaci�n, informaci�n suficiente para poder hablar de ella durante la planificaci�n del producto. Cara caracter�stica tiene tambi�n un conjunto de valores de planificaci�n que son incluidos: 

	\begin{itemize}
		\item{Prioridad con la que se ir�n desarrollando (muy alta, alta, media, baja o muy baja)}
		\item{Estado (aceptado, planificado, en desarrollo, finalizada postergada o rechazada)}
		\item{Coste estimado de implementaci�n (en t�rminos de tipos de recursos y horas-persona)}
		\item{Nivel de riesgo asociado a la implementaci�n de la caracter�stica (cr�tico, significativo o rutinario)}
	\end{itemize}
	
	Las categor�as son: \\
		Gesti�n de entrenadores\\
			Acceso y Autentificaci�n\\
		Gesti�n de nadadores\\
		Gesti�n de entrenamientos\\
			Planificaci�n\\
			Tabla de entrenamientos\\
		Gesti�n de competiciones\\
		Diario del entrenador\\

\subsection{En relaci�n a la gesti�n de entrenadores} % (fold)

	\begin{center}
	\begin{tabulary}{15cm}{|L|}
		\hline
			\bf{LC-A1. Acceso y autentificaci�n} \\
		\hline
			Esto es un ejemplo de descripci�n de la caracter�stica seleccionada donde se ocupa m�s de la cuenta debido a que es asi.\\
		\hline
			\it{Prioridad} \\
		\hline
			\it{Estado} \\
		\hline
			\it{Coste} \\
		\hline
			\it{Riesgo} \\
		\hline
	\end{tabulary}
	\end{center}
	
% subsection en_relaci�n_a_los_entrenadores (end)
	
% section lista_de_caracter�sticas (end)



	
	% Referencias bibliograficas
	\nocite{*}	% Se usa para indicar en la bibliografia las referencias no citadas.
	\bibliography{bibliografia}
	\bibliographystyle{plain}

\end{document}
