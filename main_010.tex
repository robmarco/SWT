\chapter { Captura de requisitos \label{ch:captura_requisitos}}

En este apartado se describe de forma detallada cada una de las funcionalidades que pueden componer el proyecto, con el fin de guiar el desarrollo hacia el sistema correcto. El proceso estar� apoyado en una lista de caracter�sticas y en el modelo del dominio.

%	-------------------------------------
% 	Secci�n para lista de caracter�sticas
%	-------------------------------------
\section{Lista de Caracter�sticas} % (fold)
	\label{sec:lista_de_caracter�sticas}

	Cada caracter�stica tiene un nombre corto y una breve explicaci�n, informaci�n suficiente para poder hablar de ella durante la planificaci�n del producto. Cara caracter�stica tiene tambi�n un conjunto de valores de planificaci�n que son incluidos: 

	\begin{itemize}
		\item{Prioridad con la que se ir�n desarrollando (muy alta, alta, media, baja o muy baja)}
		\item{Estado (aceptado, planificado, en desarrollo, finalizada postergada o rechazada)}
		\item{Coste estimado de implementaci�n (en t�rminos de tipos de recursos y horas-persona)}
		\item{Nivel de riesgo asociado a la implementaci�n de la caracter�stica (cr�tico, significativo o rutinario)}
	\end{itemize}
	
	Las categor�as son: \\
		Gesti�n de entrenadores\\
			Acceso y Autentificaci�n\\
		Gesti�n de nadadores\\
		Gesti�n de entrenamientos\\
			Planificaci�n\\
			Tabla de entrenamientos\\
		Gesti�n de competiciones\\
		Diario del entrenador\\

\subsection{En relaci�n a la gesti�n de entrenadores} % (fold)

	\begin{center}
	\begin{tabulary}{15cm}{|L|}
		\hline
			\bf{LC-A1. Acceso y autentificaci�n} \\
		\hline
			Esto es un ejemplo de descripci�n de la caracter�stica seleccionada donde se ocupa m�s de la cuenta debido a que es asi.\\
		\hline
			\it{Prioridad} \\
		\hline
			\it{Estado} \\
		\hline
			\it{Coste} \\
		\hline
			\it{Riesgo} \\
		\hline
	\end{tabulary}
	\end{center}
	
% subsection en_relaci�n_a_los_entrenadores (end)
	
% section lista_de_caracter�sticas (end)


