\chapter{ Introducci�n }

\section {Prueba}

Esto es una prueba para probar que funcionen correctamente las tildes con la funci�n espec�fica.
\begin{center}
\begin{tabulary}{15cm}{|L|}
	\hline
		{\bf LC-A1. A�adir nombre de usuario} \\
	\hline
		Esto es un ejemplo de descripci�n de la caracter�stica seleccionada donde se ocupa m�s de la cuenta debido a que es asi.\\
	\hline
		{\it Prioridad} \\
	\hline
		{\it Riesgo} \\
	\hline
\end{tabulary}
\end{center}

\begin{center}
\begin{tabulary}{15cm}{|L|}
	\hline
		{\bf LC-A1. A�adir nombre de usuario} \\
	\hline
		Esto es un ejemplo de descripci�n de la caracter�stica seleccionada donde se ocupa m�s de la cuenta debido a que es asi.\\
	\hline
		{\it Prioridad} \\
	\hline
		{\it Riesgo} \\
	\hline
\end{tabulary}
\end{center}



%\begin{sidewaystable}
%	\begin{center}
%		\begin{tabulary}{22cm}{|p{2cm}|p{3cm}|p{14cm}|p{2cm}|} 
%			\hline 
%				{\bf C�digo} & 
%				{\bf Nombre} & 
%				{\bf Descripci�n} &
%				{\bf Prioridad} \\
%			\hline
%				LC-A.1 &
%				Esto es una prueba de nombre largo &
%				La aplicaci�n debe permitir al usuario navegar, de forma
%				suave y en tiempo real, a trav�s del terreno. Durante la
%				navegaci�n el usuario podr� acelerar o frenar, as� como girar en cualquier direcci�n. &
%				Alta \\
%			\hline	
%				LC-A.1 &
%				Navegar &
%				La aplicaci�n debe permitir al usuario navegar, de forma
%				suave y en tiempo real, a trav�s del terreno. Durante la
%				navegaci�n el usuario podr� acelerar o frenar, as� como girar en cualquier direcci�n. &
%				Alta \\
%			\hline
%		\end{tabulary}
%	\end{center}
%\end{sidewaystable}


